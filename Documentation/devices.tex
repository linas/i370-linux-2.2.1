\documentstyle{article}
% $Id: devices.tex,v 1.1 1999/02/08 06:21:42 linas Exp $
% ---------------------------------------------------------------------------
% Adopt somewhat reasonable margins, so it doesn't take a million
% pages to print... :-)  If you're actually putting this in print, you
% may wish to change these.
\oddsidemargin=0in
\textwidth=6.5in
\topmargin=0in
\headheight=0.5in
\headsep=0.25in
\textheight=7.5in
\footskip=0.75in
\footheight=0.5in
%
\begin{document}
\newcommand{\file}{\tt}			% Style to use for a filename
\newcommand{\url}{\it}		        % Style to use for an URL
\newcommand{\hex}{\tt}			% Style to use for a hex number
\newcommand{\ud}{(Under development)}	% Abbreviation
\newcommand{\1}{\({}^1\)}
\newcommand{\2}{\({}^2\)}
\newcommand{\3}{\({}^3\)}
\newcommand{\4}{\({}^4\)}
\newlength{\dig}
\settowidth{\dig}{0}			% Get width of digits
\newcommand{\num}[2]{\makebox[#1\dig][r]{#2}}
\newcommand{\major}[4]{\num{3}{#1}#2 \> #3 \> #4 \\}
\newcommand{\minor}[3]{\> \> \num{3}{#1} \> {\file #2} \> #3 \\}
\newcommand{\minordots}{\> \> \> \dots \\}
\newenvironment{devicelist}%
 {\begin{tabbing}%
000--000 \= blockxxx \= 000 \= {\file /dev/input/keyboardxxx} \= foo \kill}%
 {\end{tabbing}}
\newcommand{\link}[4]{{\file #1} \> {\file #2} \> #3 \> #4 \\}
\newcommand{\vlink}[4]{{\file #1} \> {\em #2 \/} \> #3 \> #4 \\}
\newcommand{\node}[3]{{\file #1} \> #2 \> #3 \\}
\newcommand{\tum}{$''$}
\newenvironment{nodelist}%
 {\begin{tabbing}%
{\file /dev/crambamboli} \= {\file /proc/self/fd/99} \= symbolicxxx \=
foo \kill}%
 {\end{tabbing}}
%
% If you reformat this document, *please* make sure this information
% gets included!  This list changes frequently, so it is crucial to
% know the date of the revision.
%
\title{{\bf Linux Allocated Devices}}
\author{Maintained by H. Peter Anvin $<$hpa@zytor.com$>$}
\date{Last revised: August 10, 1998}
\maketitle
%
\noindent
This list is the Linux Device List, the official registry of allocated
device numbers and {\file /dev} directory nodes for the Linux
operating system.

The latest version of this list is included with the Linux kernel
sources in \LaTeX\ and ASCII form.  It is also available separately
from {\url ftp://ftp.kernel.org/pub/linux/docs/device-list/}.  In case
of discrepancy between the text and \LaTeX\ versions, the \LaTeX\
version is authoritative.

This document is included by reference into the Linux Filesystem
Standard (FSSTND).  The FSSTND is available from
{\url ftp://tsx-11.mit.edu/pub/linux/docs/linux-standards/fsstnd/}.

Allocations marked (68k/Amiga) apply to Linux/68k on the Amiga
platform only.  Allocations marked (68k/Atari) apply to Linux/68k on
the Atari platform only.

This document is in the public domain.  The author requests, however,
that semantically altered versions are not distributed without
permission of the author, assuming the author can be contacted without
an unreasonable effort.

In particular, please don't sent patches for this list to Linus, at
least not without contacting me first.

I do not have any information about these devices beyond what appears
on this list.  Any such information requests will be deleted without
reply.

\section{How to submit a device entry}

To have a major number allocated, or a minor number in situations
where that applies (e.g. busmice), please contact me with the
appropriate device information.  Also, if you have additional
information regarding any of the devices listed below, or if I have
made a mistake, I would greatly appreciate a note.

I do, however, make two requests about the nature of your report.
This is necessary for me to be able to keep this list up to date and
correct in a timely manner.  First of all, {\em please\/} include the
word ``device'' in the subject so your mail won't accidentally get
buried!  I receive hundreds of email messages a day, so mail sent with
other subjects may very well get lost in the avalanche.

Second, please include a description of the device {\em in the same
format as this list\/}.  The reason for this is that it is the only
way I have found to ensure I have all the requisite information to
publish your device and avoid conflicts.

Your cooperation is appreciated.

\section{Major numbers}

\begin{devicelist}
\major{ 0}{}{     }{Unnamed devices (e.g. non-device mounts)}
\major{ 1}{}{char }{Memory devices}
\major{  }{}{block}{RAM disk}
\major{ 2}{}{char }{Pseudo-TTY masters}
\major{  }{}{block}{Floppy disks}
\major{ 3}{}{char }{Pseudo-TTY slaves}
\major{  }{}{block}{First MFM, RLL or IDE hard disk/CD-ROM interface}
\major{ 4}{}{char }{TTY devices}
\major{ 5}{}{char }{Alternate TTY devices}
\major{ 6}{}{char }{Parallel printer devices}
\major{ 7}{}{char }{Virtual console access devices}
\major{  }{}{block}{Loopback devices}
\major{ 8}{}{block}{SCSI disk devices (0-15)}
\major{ 9}{}{char }{SCSI tape devices}
\major{  }{}{block}{Metadisk (RAID) devices}
\major{10}{}{char }{Non-serial mice, misc features}
\major{11}{}{char }{Raw keyboard device}
\major{  }{}{block}{SCSI CD-ROM devices}
\major{12}{}{char }{QIC-02 tape}
\major{  }{}{block}{MSCDEX CD-ROM callback support}
\major{13}{}{char }{PC speaker}
\major{  }{}{block}{8-bit MFM/RLL/IDE controller}
\major{14}{}{char }{Sound card}
\major{  }{}{block}{BIOS harddrive callback support}
\major{15}{}{char }{Joystick}
\major{  }{}{block}{Sony CDU-31A/CDU-33A CD-ROM}
\major{16}{}{char }{Non-SCSI scanners}
\major{  }{}{block}{GoldStar CD-ROM}
\major{17}{}{char }{Chase serial card}
\major{  }{}{block}{Optics Storage CD-ROM}
\major{18}{}{char }{Chase serial card -- alternate devices}
\major{  }{}{block}{Sanyo CD-ROM}
\major{19}{}{char }{Cyclades serial card}
\major{  }{}{block}{``Double'' compressed disk}
\major{20}{}{char }{Cyclades serial card -- alternate devices}
\major{  }{}{block}{Hitachi CD-ROM}
\major{21}{}{char }{Generic SCSI access}
\major{  }{}{block }{Acorn MFM hard drive interface}
\major{22}{}{char }{Digiboard serial card}
\major{  }{}{block}{Second IDE hard disk/CD-ROM interface}
\major{23}{}{char }{Digiboard serial card -- alternate devices}
\major{  }{}{block}{Mitsumi proprietary CD-ROM}
\major{24}{}{char }{Stallion serial card}
\major{  }{}{block}{Sony CDU-535 CD-ROM}
\major{25}{}{char }{Stallion serial card -- alternate devices}
\major{  }{}{block}{First Matsushita (Panasonic/SoundBlaster) CD-ROM}
\major{26}{}{char }{Quanta WinVision frame grabber}
\major{  }{}{block}{Second Matsushita (Panasonic/SoundBlaster) CD-ROM}
\major{27}{}{char }{QIC-117 tape}
\major{  }{}{block}{Third Matsushita (Panasonic/SoundBlaster) CD-ROM}
\major{28}{}{char }{Stallion serial card -- card programming}
\major{  }{}{char }{Atari SLM ACSI laser printer (68k/Atari)}
\major{  }{}{block}{Fourth Matsushita (Panasonic/SoundBlaster) CD-ROM}
\major{  }{}{block}{ACSI disk/CD-ROM (68k/Atari)}
\major{29}{}{char }{Universal frame buffer}
\major{  }{}{block}{Aztech/Orchid/Okano/Wearnes CD-ROM}
\major{30}{}{char }{iBCS-2 compatibility devices}
\major{  }{}{block}{Philips LMS CM-205 CD-ROM}
\major{31}{}{char }{MPU-401 MIDI}
\major{  }{}{block}{ROM/flash memory card}
\major{32}{}{char }{Specialix serial card}
\major{  }{}{block}{Philips LMS CM-206 CD-ROM}
\major{33}{}{char }{Specialix serial card -- alternate devices}
\major{  }{}{block}{Third IDE hard disk/CD-ROM interface}
\major{34}{}{char }{Z8530 HDLC driver}
\major{  }{}{block}{Fourth IDE hard disk/CD-ROM interface}
\major{35}{}{char }{tclmidi MIDI driver}
\major{  }{}{block}{Slow memory ramdisk}
\major{36}{}{char }{Netlink support}
\major{  }{}{block}{MCA ESDI hard disk}
\major{37}{}{char }{IDE tape}
\major{  }{}{block}{Zorro II ramdisk}
\major{38}{}{char }{Myricom PCI Myrinet board}
\major{  }{}{block}{Reserved for Linux/AP+}
\major{39}{}{char }{ML-16P experimental I/O board}
\major{  }{}{block}{Reserved for Linux/AP+}
\major{40}{}{char }{Matrox Meteor frame grabber}
\major{  }{}{block}{Syquest EZ135 parallel port removable drive}
\major{41}{}{char }{Yet Another Micro Monitor}
\major{  }{}{block}{MicroSolutions BackPack parallel port CD-ROM}
\major{42}{}{}{Demo/sample use}
\major{43}{}{char }{isdn4linux virtual modem}
\major{  }{}{block}{Network block devices}
\major{44}{}{char }{isdn4linux virtual modem -- alternate devices}
\major{  }{}{block}{Flash Translation Layer (FTL) filesystems}
\major{45}{}{char }{isdn4linux ISDN BRI driver}
\major{  }{}{block}{Parallel port IDE disk devices}
\major{46}{}{char }{Comtrol Rocketport serial card}
\major{  }{}{block}{Parallel port ATAPI CD-ROM devices}
\major{47}{}{char }{Comtrol Rocketport serial card -- alternate devices}
\major{  }{}{block}{Parallel port ATAPI disk devices}
\major{48}{}{char }{SDL RISCom serial card}
\major{  }{}{block}{Reserved for Mylex DAC960 PCI RAID Controller}
\major{49}{}{char }{SDL RISCom serial card -- alternate devices}
\major{  }{}{block}{Reserved for Mylex DAC960 PCI RAID Controller}
\major{50}{}{char}{Reserved for GLINT}
\major{  }{}{block}{Reserved for Mylex DAC960 PCI RAID Controller}
\major{51}{}{char }{Baycom radio modem}
\major{  }{}{block}{Reserved for Mylex DAC960 PCI RAID Controller}
\major{52}{}{char }{Spellcaster DataComm/BRI ISDN card}
\major{  }{}{block}{Reserved for Mylex DAC960 PCI RAID Controller}
\major{53}{}{char }{BDM interface for remote debugging MC683xx microcontrollers}
\major{  }{}{block}{Reserved for Mylex DAC960 PCI RAID Controller}
\major{54}{}{char }{Electrocardiognosis Holter serial card}
\major{  }{}{block}{Reserved for Mylex DAC960 PCI RAID Controller}
\major{55}{}{char }{DSP56001 digital signal processor}
\major{  }{}{block}{Reserved for Mylex DAC960 PCI RAID Controller}
\major{56}{}{char }{Apple Desktop Bus}
\major{  }{}{block}{Fifth IDE hard disk/CD-ROM interface}
\major{57}{}{char }{Hayes ESP serial card}
\major{  }{}{block}{Sixth IDE hard disk/CD-ROM interface}
\major{58}{}{char }{Hayes ESP serial card -- alternate devices}
\major{  }{}{block}{Reserved for logical volume manager}
\major{59}{}{char }{sf firewall package}
\major{60}{--63}{}{Local/experimental use}
\major{64}{}{char }{ENskip kernel encryption package}
\major{65}{}{char }{Sundance ``plink'' Transputer boards}
\major{  }{}{block}{SCSI disk devices (16-31)}
\major{66}{}{char }{YARC PowerPC PCI coprocessor card}
\major{  }{}{block}{SCSI disk devices (32-47)}
\major{67}{}{char }{Coda network filesystem}
\major{  }{}{block}{SCSI disk devices (48-63)}
\major{68}{}{char }{CAPI 2.0 interface}
\major{  }{}{block}{SCSI disk devices (64-79)}
\major{69}{}{char }{MA16 numeric accelerator card}
\major{  }{}{block}{SCSI disk devices (80-95)}
\major{70}{}{char }{SpellCaster Protocol Services Interface}
\major{  }{}{block}{SCSI disk devices (96-111)}
\major{71}{}{char }{Computone IntelliPort II serial card}
\major{  }{}{block}{SCSI disk devices (112-127)}
\major{72}{}{char }{Computone IntelliPort II serial card -- alternate devices}
\major{73}{}{char }{Computone IntelliPort II serial card -- control devices}
\major{74}{}{char }{SCI bridge}
\major{75}{}{char }{Specialix IO8+ serial card}
\major{76}{}{char }{Specialix IO8+ serial card -- alternate devices}
\major{77}{}{char }{ComScire Quantum Noise Generator}
\major{78}{}{char }{PAM Software's multimodem boards}
\major{79}{}{char }{PAM Software's multimodem boards -- alternate devices}
\major{80}{}{char }{Photometrics AT200 CCD camera}
\major{81}{}{char }{video4linux}
\major{82}{}{char }{WiNRADiO communications receiver card}
\major{83}{}{char }{Teletext/videotext interfaces}
\major{84}{}{char }{Ikon 1011[57] Versatec Greensheet Interface}
\major{85}{}{char }{Linux/SGI shared memory input queue}
\major{86}{}{char }{SCSI media changer}
\major{87}{}{char }{Sony Control-A1 stereo control bus}
\major{88}{}{char }{COMX synchronous serial card}
\major{89}{}{char }{I$^2$C bus interface}
\major{90}{}{char }{Memory Technology Device (RAM, ROM, Flash)}
\major{91}{}{char }{CAN-Bus controller}
\major{92}{}{char }{Reserved for ith Kommunikationstechnik MIC ISDN card}
\major{93}{}{char }{IBM Smart Capture Card frame grabber}
\major{94}{}{char }{miroVIDEO DC10/30 capture/playback device}
\major{95}{}{char }{IP filter}
\major{96}{}{char }{Parallel port ATAPI tape devices}
\major{97}{}{char }{Parallel port generic ATAPI interface}
\major{98}{}{char }{Control and Mesurement Device (comedi)}
\major{99}{}{char }{Raw parallel ports}
\major{100}{}{char }{POTS (analogue telephone) A/B port}
\major{101}{}{char }{Motorola DSP 56xxx board}
\major{102}{}{char }{Philips SAA5249 Teletext signal decoder}
\major{103}{}{char }{Arla network file system}
\major{104}{}{char }{Flash BIOS support}
\major{105}{}{char }{Comtrol VS-1000 serial card}
\major{106}{}{char }{Comtrol VS-1000 serial card -- alternate devices}
\major{107}{}{char }{3Dfx Voodoo Graphics device}
\major{108}{}{char }{Device independent PPP interface}
\major{109}{}{char }{Reserved for logical volume manager}
\major{110}{}{char }{miroMEDIA Surround board}
\major{111}{--119}{}{Unallocated}
\major{120}{--127}{}{Local/experimental use}
\major{128}{--135}{char }{Unix98 PTY masters}
\major{136}{--143}{char }{Unix98 PTY slaves}
\major{144}{--239}{}{Unallocated}
\major{240}{--254}{}{Local/experimental use}
\major{255}{}{}{Reserved}
\end{devicelist}

\section{Minor numbers}

\begin{devicelist}
\major{ 0}{}{}{Unnamed devices (e.g. non-device mounts)}
	\minor{0}{}{reserved as null device number}
\end{devicelist}

\begin{devicelist}
\major{ 1}{}{char}{Memory devices}
	\minor{1}{/dev/mem}{Physical memory access}
	\minor{2}{/dev/kmem}{Kernel virtual memory access}
	\minor{3}{/dev/null}{Null device}
	\minor{4}{/dev/port}{I/O port access}
	\minor{5}{/dev/zero}{Null byte source}
	\minor{6}{/dev/core}{OBSOLETE -- should be a link to {\file /proc/kcore}}
	\minor{7}{/dev/full}{Returns ENOSPC on write}
	\minor{8}{/dev/random}{Nondeterministic random number generator}
	\minor{9}{/dev/urandom}{Less secure, but faster random number generator}
\\
\major{}{}{block}{RAM disk}
	\minor{0}{/dev/ram0}{First RAM disk}
	\minordots
	\minor{7}{/dev/ram7}{Eighth RAM disk}
	\minor{250}{/dev/initrd}{Initial RAM disk}
\end{devicelist}

\noindent
Earlier kernels had {\file /dev/ramdisk} (1, 1) here.  {\file /dev/initrd}
refers to a RAM disk which was preloaded by the boot loader.

\begin{devicelist}
\major{ 2}{}{char}{Pseudo-TTY masters}
	\minor{0}{/dev/ptyp0}{First PTY master}
	\minor{1}{/dev/ptyp1}{Second PTY master}
	\minordots
	\minor{255}{/dev/ptyef}{256th PTY master}
\end{devicelist}

\noindent
Pseudo-TTY's are named as follows:
\begin{itemize}
\item Masters are {\file pty}, slaves are {\file tty};
\item the fourth letter is one of {\file pqrstuvwxyzabcde} indicating
the 1st through 16th series of 16 pseudo-ttys each, and
\item the fifth letter is one of {\file 0123456789abcdef} indicating
the position within the series.
\end{itemize}

\noindent
These are the old-style (BSD) PTY devices; Unix98 devices are on major
128 and above and use the PTY master multiplex ({\file /dev/ptmx}) to
acquire a PTY on demand.

\begin{devicelist}
\major{}{}{block}{Floppy disks}
	\minor{0}{/dev/fd0}{Controller 1, drive 1 autodetect}
	\minor{1}{/dev/fd1}{Controller 1, drive 2 autodetect}
	\minor{2}{/dev/fd2}{Controller 1, drive 3 autodetect}
	\minor{3}{/dev/fd3}{Controller 1, drive 4 autodetect}
	\minor{128}{/dev/fd4}{Controller 2, drive 1 autodetect}
	\minor{129}{/dev/fd5}{Controller 2, drive 2 autodetect}
	\minor{130}{/dev/fd6}{Controller 2, drive 3 autodetect}
	\minor{131}{/dev/fd7}{Controller 2, drive 4 autodetect}
\\
\major{}{}{}{To specify format, add to the autodetect device number}
	\minor{  0}{/dev/fd?}{Autodetect format}
	\minor{}{}{}
	\minor{  4}{/dev/fd?d360}{5.25\tum\ \num{4}{360}K in a \num{4}{360}K drive\1}
	\minor{ 20}{/dev/fd?h360}{5.25\tum\ \num{4}{360}K in a 1200K drive\1}
	\minor{ 48}{/dev/fd?h410}{5.25\tum\ \num{4}{410}K in a 1200K drive}
	\minor{ 64}{/dev/fd?h420}{5.25\tum\ \num{4}{420}K in a 1200K drive}
	\minor{ 24}{/dev/fd?h720}{5.25\tum\ \num{4}{720}K in a 1200K drive}
	\minor{ 80}{/dev/fd?h880}{5.25\tum\ \num{4}{880}K in a 1200K drive\1}
	\minor{  8}{/dev/fd?h1200}{5.25\tum\ 1200K in a 1200K drive\1}
	\minor{ 40}{/dev/fd?h1440}{5.25\tum\ 1440K in a 1200K drive\1}
	\minor{ 56}{/dev/fd?h1476}{5.25\tum\ 1476K in a 1200K drive}
	\minor{ 72}{/dev/fd?h1494}{5.25\tum\ 1494K in a 1200K drive}
	\minor{ 92}{/dev/fd?h1600}{5.25\tum\ 1600K in a 1200K drive\1}
	\minor{}{}{}
	\minor{ 12}{/dev/fd?u360}{3.5\tum\ \num{4}{360}K Double Density\2}
	\minor{ 16}{/dev/fd?u720}{3.5\tum\ \num{4}{720}K Double Density\1}
	\minor{120}{/dev/fd?u800}{3.5\tum\ \num{4}{800}K Double Density\2}
	\minor{ 52}{/dev/fd?u820}{3.5\tum\ \num{4}{820}K Double Density}
	\minor{ 68}{/dev/fd?u830}{3.5\tum\ \num{4}{830}K Double Density}
	\minor{ 84}{/dev/fd?u1040}{3.5\tum\ 1040K Double Density\1}
	\minor{ 88}{/dev/fd?u1120}{3.5\tum\ 1120K Double Density\1}
	\minor{ 28}{/dev/fd?u1440}{3.5\tum\ 1440K High Density\1}
	\minor{124}{/dev/fd?u1600}{3.5\tum\ 1600K High Density\1}
	\minor{ 44}{/dev/fd?u1680}{3.5\tum\ 1680K High Density\3}
	\minor{ 60}{/dev/fd?u1722}{3.5\tum\ 1722K High Density}
	\minor{ 76}{/dev/fd?u1743}{3.5\tum\ 1743K High Density}
	\minor{ 96}{/dev/fd?u1760}{3.5\tum\ 1760K High Density}
	\minor{116}{/dev/fd?u1840}{3.5\tum\ 1840K High Density\3}
	\minor{100}{/dev/fd?u1920}{3.5\tum\ 1920K High Density\1}
	\minor{ 32}{/dev/fd?u2880}{3.5\tum\ 2880K Extra Density\1}
	\minor{104}{/dev/fd?u3200}{3.5\tum\ 3200K Extra Density}
	\minor{108}{/dev/fd?u3520}{3.5\tum\ 3520K Extra Density}
	\minor{112}{/dev/fd?u3840}{3.5\tum\ 3840K Extra Density\1}
	\minor{}{}{}
	\minor{36}{/dev/fd?CompaQ}{Compaq 2880K drive; probably obsolete}
\\
\major{}{}{}{\1 Autodetectable format}
\major{}{}{}{\2 Autodetectable format in a Double Density (720K) drive only}
\major{}{}{}{\3 Autodetectable format in a High Density (1440K) drive only}
\end{devicelist}

NOTE: The letter in the device name ({\file d}, {\file q}, {\file h}
or {\file u}) signifies the type of drive supported: 5.25\tum\ Double
Density ({\file d}), 5.25\tum\ Quad Density ({\file q}), 5.25\tum\
High Density ({\file h}) or 3.5\tum\ (any type, {\file u}).  The
capital letters {\file D}, {\file H}, or {\file E} for the 3.5\tum\
models have been deprecated, since the drive type is insignificant for
these devices.

\begin{devicelist}
\major{ 3}{}{char}{Pseudo-TTY slaves}
	\minor{0}{/dev/ttyp0}{First PTY slave}
	\minor{1}{/dev/ttyp1}{Second PTY slave}
	\minordots
	\minor{255}{/dev/ttyef}{256th PTY slave}
\end{devicelist}

\noindent
These are the old-style (BSD) PTY devices; Unix98 devices are on major
136 and above.

\begin{devicelist}
\major{}{}{block}{First MFM, RLL and IDE hard disk/CD-ROM interface}
	\minor{0}{/dev/hda}{Master: whole disk (or CD-ROM)}
	\minor{64}{/dev/hdb}{Slave: whole disk (or CD-ROM)}
\\
\major{}{}{}{For partitions, add to the whole disk device number}
	\minor{0}{/dev/hd?}{Whole disk}
	\minor{1}{/dev/hd?1}{First partition}
	\minor{2}{/dev/hd?2}{Second partition}
	\minordots
	\minor{63}{/dev/hd?63}{63rd partition}
\end{devicelist}

\noindent
For MS-DOS style partition tables (typically used by Linux/i386 and
sometimes on Linux/Alpha), partitions 1-4 are the primary partitions,
partitions 5 and up are logical partitions.  For other partitioning
schemes, the meaning of the numbers vary.

\begin{devicelist}
\major{ 4}{}{char }{TTY devices}
	\minor{0}{/dev/tty0}{Current virtual console}
	\minor{1}{/dev/tty1}{First virtual console}
	\minordots
	\minor{63}{/dev/tty63}{63rd virtual console}
	\minor{64}{/dev/ttyS0}{First serial port}
	\minordots
	\minor{127}{/dev/ttyS63}{64th serial port}
	\minor{128}{/dev/ptyp0}{OBSOLETE}
	\minordots
	\minor{191}{/dev/ptysf}{OBSOLETE}
	\minor{192}{/dev/ttyp0}{OBSOLETE}
	\minordots
	\minor{255}{/dev/ttysf}{OBSOLETE}
\end{devicelist}

\noindent
Older versions of the Linux kernel used this major number for BSD PTY
devices.  As of Linux 2.1.115, this is no longer supported.  Use major
numbers 2 and 3.

\begin{devicelist}
\major{ 5}{}{char }{Alternate TTY devices}
	\minor{0}{/dev/tty}{Current TTY device}
	\minor{1}{/dev/console}{System console}
	\minor{2}{/dev/ptmx}{PTY master multiplex}
	\minor{64}{/dev/cua0}{Callout device corresponding to {\file ttyS0}}
	\minordots
	\minor{127}{/dev/cua63}{Callout device corresponding to {\file ttyS63}}
\end{devicelist}

\noindent
(5,1) is {\file /dev/console} starting with Linux 2.1.71.  See the
section on terminal devices for more information on {\file /dev/console}.

\begin{devicelist}
\major{ 6}{}{char }{Parallel printer devices}
	\minor{0}{/dev/lp0}{First parallel printer ({\hex 0x3bc})}
	\minor{1}{/dev/lp1}{Second parallel printer ({\hex 0x378})}
	\minor{2}{/dev/lp2}{Third parallel printer ({\hex 0x278})}
\end{devicelist}

\noindent
Not all computers have the {\hex 0x3bc} parallel port, hence the
"first" printer may be either {\file /dev/lp0} or {\file /dev/lp1}.

\begin{devicelist}
\major{ 7}{}{char }{Virtual console access devices}
	\minor{0}{/dev/vcs}{Current vc text access}
	\minor{1}{/dev/vcs1}{tty1 text access}
	\minordots
	\minor{63}{/dev/vcs63}{tty63 text access}
	\minor{128}{/dev/vcsa}{Current vc text/attribute access}
	\minor{129}{/dev/vcsa1}{tty1 text/attribute access}
	\minordots
	\minor{191}{/dev/vcsa63}{tty63 text/attribute access}
\end{devicelist}

\noindent
NOTE: These devices permit both read and write access.

\begin{devicelist}
\major{  }{}{block}{Loopback devices}
	\minor{0}{/dev/loop0}{First loopback device}
	\minor{1}{/dev/loop1}{Second loopback device}
	\minordots
\end{devicelist}

\noindent
The loopback devices are used to mount filesystems not associated with
block devices.  The binding to the loopback devices is handled by
{\bf mount}(8) or {\bf losetup}(8).

\begin{devicelist}
\major{ 8}{}{block}{SCSI disk devices (0-15)}
	\minor{0}{/dev/sda}{First SCSI disk whole disk}
	\minor{16}{/dev/sdb}{Second SCSI disk whole disk}
	\minor{32}{/dev/sdc}{Third SCSI disk whole disk}
	\minordots
	\minor{240}{/dev/sdp}{Sixteenth SCSI disk whole disk}
\end{devicelist}

\noindent
Partitions are handled in the same way as for IDE disks (see major
number 3) except that the partition limit is 15 rather than 63 per
disk.

\begin{devicelist}
\major{ 9}{}{char }{SCSI tape devices}
	\minor{0}{/dev/st0}{First SCSI tape, mode 0}
	\minor{1}{/dev/st1}{Second SCSI tape, mode 0}
	\minordots
	\minor{32}{/dev/st0l}{First SCSI tape, mode 1}
	\minor{33}{/dev/st1l}{Second SCSI tape, mode 1}
	\minordots
	\minor{64}{/dev/st0m}{First SCSI tape, mode 2}
	\minor{65}{/dev/st1m}{Second SCSI tape, mode 2}
	\minordots
	\minor{96}{/dev/st0a}{First SCSI tape, mode 3}
	\minor{97}{/dev/st1a}{Second SCSI tape, mode 4}
	\minordots
	\minor{128}{/dev/nst0}{First SCSI tape, mode 0, no rewind}
	\minor{129}{/dev/nst1}{Second SCSI tape, mode 0, no rewind}
	\minordots
	\minor{160}{/dev/nst0l}{First SCSI tape, mode 1, no rewind}
	\minor{161}{/dev/nst1l}{Second SCSI tape, mode 1, no rewind}
	\minordots
	\minor{192}{/dev/nst0m}{First SCSI tape, mode 2, no rewind}
	\minor{193}{/dev/nst1m}{Second SCSI tape, mode 2, no rewind}
	\minordots
	\minor{224}{/dev/nst0a}{First SCSI tape, mode 3, no rewind}
	\minor{225}{/dev/nst1a}{Second SCSI tape, mode 3, no rewind}
	\minordots
\end{devicelist}

\noindent
``No rewind'' refers to the omission of the default automatic rewind
on device close.  The {\file MTREW} or {\file MTOFFL} ioctl()s can be
used to rewind the tape regardless of the device used to access it.

\begin{devicelist}
\major{  }{}{block}{Metadisk (RAID) devices}
	\minor{0}{/dev/md0}{First metadisk group}
	\minor{1}{/dev/md1}{Second metadisk group}
	\minordots
\end{devicelist}

\noindent
The metadisk driver is used to span a filesystem across multiple
physical disks.

\begin{devicelist}
\major{10}{}{char }{Non-serial mice, misc features}
	\minor{0}{/dev/logibm}{Logitech bus mouse}
	\minor{1}{/dev/psaux}{PS/2-style mouse port}
	\minor{2}{/dev/inportbm}{Microsoft Inport bus mouse}
	\minor{3}{/dev/atibm}{ATI XL bus mouse}
	\minor{4}{/dev/jbm}{J-mouse}
	\minor{4}{/dev/amigamouse}{Amiga mouse (68k/Amiga)}
	\minor{5}{/dev/atarimouse}{Atari mouse}
	\minor{6}{/dev/sunmouse}{Sun mouse}
	\minor{7}{/dev/amigamouse1}{Second Amiga mouse}
	\minor{8}{/dev/smouse}{Simple serial mouse driver}
	\minor{9}{/dev/pc110pad}{IBM PC-110 digitizer pad}
	\minor{128}{/dev/beep}{Fancy beep device}
	\minor{129}{/dev/modreq}{Kernel module load request}
	\minor{130}{/dev/watchdog}{Watchdog timer port}
	\minor{131}{/dev/temperature}{Machine internal temperature}
	\minor{132}{/dev/hwtrap}{Hardware fault trap}
	\minor{133}{/dev/exttrp}{External device trap}
	\minor{134}{/dev/apm\_bios}{Advanced Power Management BIOS}
	\minor{135}{/dev/rtc}{Real Time Clock}
	\minor{139}{/dev/openprom}{SPARC OpenBoot PROM}
	\minor{140}{/dev/relay8}{Berkshire Products Octal relay card}
	\minor{141}{/dev/relay16}{Berkshire Products ISO-16 relay card}
	\minor{142}{/dev/msr}{x86 model specific registers}
	\minor{143}{/dev/pciconf}{PCI configuration space}
	\minor{144}{/dev/nvram}{Non-volatile configuration RAM}
	\minor{145}{/dev/hfmodem}{Soundcard shortwave modem control}
	\minor{146}{/dev/graphics}{Linux/SGI graphics device}
	\minor{147}{/dev/opengl}{Linux/SGI OpenGL pipe}
	\minor{148}{/dev/gfx}{Linux/SGI graphics effects device}
	\minor{149}{/dev/input/mouse}{Linux/SGI Irix emulation mouse}
	\minor{150}{/dev/input/keyboard}{Linux/SGI Irix emulation keyboard}
	\minor{151}{/dev/led}{Front panel LEDs}
	\minor{153}{/dev/mergemem}{Memory merge device}
	\minor{154}{/dev/pmu}{Macintosh PowerBook power manager}
\end{devicelist}

\begin{devicelist}
\major{11}{}{char }{Raw keyboard device}
	\minor{0}{/dev/kbd}{Raw keyboard device}
\end{devicelist}

\noindent
The raw keyboard device is used on Linux/SPARC only.

\begin{devicelist}
\major{  }{}{block}{SCSI CD-ROM devices}
	\minor{0}{/dev/sr0}{First SCSI CD-ROM}
	\minor{1}{/dev/sr1}{Second SCSI CD-ROM}
	\minordots
\end{devicelist}

\noindent
The prefix {\file /dev/scd} instead of {\file /dev/sr} has been used
as well, and might make more sense.

\begin{devicelist}
\major{12}{}{char }{QIC-02 tape}
	\minor{2}{/dev/ntpqic11}{QIC-11, no rewind-on-close}
	\minor{3}{/dev/tpqic11}{QIC-11, rewind-on-close}
	\minor{4}{/dev/ntpqic24}{QIC-24, no rewind-on-close}
	\minor{5}{/dev/tpqic24}{QIC-24, rewind-on-close}
	\minor{6}{/dev/ntpqic120}{QIC-120, no rewind-on-close}
	\minor{7}{/dev/tpqic120}{QIC-120, rewind-on-close}
	\minor{8}{/dev/ntpqic150}{QIC-150, no rewind-on-close}
	\minor{9}{/dev/tpqic150}{QIC-150, rewind-on-close}
\end{devicelist}

\noindent
The device names specified are proposed -- if there are ``standard''
names for these devices, please let me know.

\begin{devicelist}
\major{  }{}{block}{MSCDEX CD-ROM callback support}
	\minor{0}{/dev/dos\_cd0}{First MSCDEX CD-ROM}
	\minor{1}{/dev/dos\_cd1}{Second MSCDEX CD-ROM}
	\minordots
\end{devicelist}

\begin{devicelist}
\major{13}{}{char }{PC speaker}
	\minor{0}{/dev/pcmixer}{Emulates {\file /dev/mixer}}
	\minor{3}{/dev/pcsp}{Emulates {\file /dev/dsp} (8-bit)}
	\minor{4}{/dev/pcaudio}{Emulates {\file /dev/audio}}
	\minor{5}{/dev/pcsp16}{Emulates {\file /dev/dsp} (16-bit)}
\\
\major{  }{}{block}{8-bit MFM/RLL/IDE controller}
	\minor{0}{/dev/xda}{First XT disk whole disk}
	\minor{64}{/dev/xdb}{Second XT disk whole disk}
\end{devicelist}

\noindent
Partitions are handled in the same way as for IDE disks (see major
number 3).

\begin{devicelist}
\major{14}{}{char }{Sound card}
	\minor{0}{/dev/mixer}{Mixer control}
	\minor{1}{/dev/sequencer}{Audio sequencer}
	\minor{2}{/dev/midi00}{First MIDI port}
	\minor{3}{/dev/dsp}{Digital audio}
	\minor{4}{/dev/audio}{Sun-compatible digital audio}
	\minor{6}{/dev/sndstat}{Sound card status information}
	\minor{8}{/dev/sequencer2}{Sequencer -- alternate device}
	\minor{16}{/dev/mixer1}{Second soundcard mixer control}
	\minor{17}{/dev/patmgr0}{Sequencer patch manager}
	\minor{18}{/dev/midi01}{Second MIDI port}
	\minor{19}{/dev/dsp1}{Second soundcard digital audio}
	\minor{20}{/dev/audio1}{Second soundcard Sun digital audio}
	\minor{33}{/dev/patmgr1}{Sequencer patch manager}
	\minor{34}{/dev/midi02}{Third MIDI port}
	\minor{50}{/dev/midi03}{Fourth MIDI port}
\\
\major{  }{}{block}{BIOS harddrive callback support}
	\minor{0}{/dev/dos\_hda}{First BIOS harddrive whole disk}
	\minor{64}{/dev/dos\_hdb}{Second BIOS harddrive whole disk}
	\minor{128}{/dev/dos\_hdc}{Third BIOS harddrive whole disk}
	\minor{192}{/dev/dos\_hdd}{Fourth BIOS harddrive whole disk}
\end{devicelist}

\noindent
Partitions are handled in the same way as for IDE disks (see major
number 3).

\begin{devicelist}
\major{15}{}{char }{Joystick}
	\minor{0}{/dev/js0}{First analog joystick}
	\minor{1}{/dev/js1}{Second analog joystick}
	\minordots
	\minor{128}{/dev/djs0}{First digital joystick}
	\minor{129}{/dev/djs1}{Second digital joystick}
	\minordots
\\
\major{  }{}{block}{Sony CDU-31A/CDU-33A CD-ROM}
	\minor{0}{/dev/sonycd}{Sony CDU-31A CD-ROM}
\end{devicelist}

\begin{devicelist}
\major{16}{}{char }{Non-SCSI scanners}
	\minor{0}{/dev/gs4500}{Genius 4500 handheld scanner}
\\
\major{  }{}{block}{GoldStar CD-ROM}
	\minor{0}{/dev/gscd}{GoldStar CD-ROM}
\end{devicelist}

\begin{devicelist}
\major{17}{}{char }{Chase serial card}
	\minor{0}{/dev/ttyH0}{First Chase port}
	\minor{1}{/dev/ttyH1}{Second Chase port}
	\minordots
\\
\major{  }{}{block}{Optics Storage CD-ROM}
	\minor{0}{/dev/optcd}{Optics Storage CD-ROM}
\end{devicelist}

\begin{devicelist}
\major{18}{}{char }{Chase serial card -- alternate devices}
	\minor{0}{/dev/cuh0}{Callout device corresponding to {\file ttyH0}}
	\minor{1}{/dev/cuh1}{Callout device corresponding to {\file ttyH1}}
	\minordots
\\
\major{  }{}{block}{Sanyo CD-ROM}
	\minor{0}{/dev/sjcd}{Sanyo CD-ROM}
\end{devicelist}

\begin{devicelist}
\major{19}{}{char }{Cyclades serial card}
	\minor{0}{/dev/ttyC0}{First Cyclades port}
	\minordots
	\minor{31}{/dev/ttyC31}{32nd Cyclades port}
\\
\major{  }{}{block}{``Double'' compressed disk}
	\minor{0}{/dev/double0}{First compressed disk}
	\minordots
	\minor{7}{/dev/double7}{Eighth compressed disk}
	\minor{128}{/dev/cdouble0}{Mirror of first compressed disk}
	\minordots
	\minor{135}{/dev/cdouble7}{Mirror of eighth compressed disk}
\end{devicelist}

\noindent
See the Double documentation for an explanation of the ``mirror'' devices.

\begin{devicelist}
\major{20}{}{char }{Cyclades serial card -- alternate devices}
	\minor{0}{/dev/cub0}{Callout device corresponding to {\file ttyC0}}
	\minordots
	\minor{31}{/dev/cub31}{Callout device corresponding to {\file ttyC31}}
\\
\major{  }{}{block}{Hitachi CD-ROM}
	\minor{0}{/dev/hitcd}{Hitachi CD-ROM}
\end{devicelist}

\begin{devicelist}
\major{21}{}{char }{Generic SCSI access}
	\minor{0}{/dev/sg0}{First generic SCSI device}
	\minor{1}{/dev/sg1}{Second generic SCSI device}
	\minordots
\end{devicelist}

\noindent
Most distributions name these {\file /dev/sga}, {\file /dev/sgb}...
This sets an unneccesary limit of 26 SCSI devices in the system, and
is counter to standard Linux device-naming practice.

\begin{devicelist}
\major{  }{}{block }{Acorn MFM hard drive interface}
	\minor{0}{/dev/mfma}{First MFM drive whole disk}
	\minor{64}{/dev/mfmb}{Second MFM drive whole disk}
\end{devicelist}

\noindent
This device is used on the ARM-based Acorn RiscPC.  Partitions are
handled the same way as for IDE disks (see major number 3).

\begin{devicelist}
\major{22}{}{char }{Digiboard serial card}
	\minor{0}{/dev/ttyD0}{First Digiboard port}
	\minor{1}{/dev/ttyD1}{Second Digiboard port}
	\minordots
\\
\major{  }{}{block}{Second IDE hard disk/CD-ROM interface}
	\minor{0}{/dev/hdc}{Master: whole disk (or CD-ROM)}
	\minor{64}{/dev/hdd}{Slave: whole disk (or CD-ROM)}
\end{devicelist}

\noindent
Partitions are handled the same way as for the first interface (see
major number 3).

\begin{devicelist}
\major{23}{}{char }{Digiboard serial card -- alternate devices}
	\minor{0}{/dev/cud0}{Callout device corresponding to {\file ttyD0}}
	\minor{1}{/dev/cud1}{Callout device corresponding to {\file ttyD1}}
	\minordots
\major{  }{}{block}{Mitsumi proprietary CD-ROM}
	\minor{0}{/dev/mcd}{Mitsumi CD-ROM}
\end{devicelist}

\begin{devicelist}\
\major{24}{}{char }{Stallion serial card}
	\minor{0}{/dev/ttyE0}{Stallion port 0 board 0}
	\minor{1}{/dev/ttyE1}{Stallion port 1 board 0}
	\minordots
	\minor{64}{/dev/ttyE64}{Stallion port 0 board 1}
	\minor{65}{/dev/ttyE65}{Stallion port 1 board 1}
	\minordots
	\minor{128}{/dev/ttyE128}{Stallion port 0 board 2}
	\minor{129}{/dev/ttyE129}{Stallion port 1 board 2}
	\minordots
	\minor{192}{/dev/ttyE192}{Stallion port 0 board 3}
	\minor{193}{/dev/ttyE193}{Stallion port 1 board 3}
	\minordots
\\
\major{  }{}{block}{Sony CDU-535 CD-ROM}
	\minor{0}{/dev/cdu535}{Sony CDU-535 CD-ROM}
\end{devicelist}

\begin{devicelist}
\major{25}{}{char }{Stallion serial card -- alternate devices}
	\minor{0}{/dev/cue0}{Callout device corresponding to {\file ttyE0}}
	\minor{1}{/dev/cue1}{Callout device corresponding to {\file ttyE1}}
	\minordots
	\minor{64}{/dev/cue64}{Callout device corresponding to {\file ttyE64}}
	\minor{65}{/dev/cue65}{Callout device corresponding to {\file ttyE65}}
	\minordots
	\minor{128}{/dev/cue128}{Callout device corresponding to {\file ttyE128}}
	\minor{129}{/dev/cue129}{Callout device corresponding to {\file ttyE129}}
	\minordots
	\minor{192}{/dev/cue192}{Callout device corresponding to {\file ttyE192}}
	\minor{193}{/dev/cue193}{Callout device corresponding to {\file ttyE193}}
	\minordots
\\
\major{  }{}{block}{First Matsushita (Panasonic/SoundBlaster) CD-ROM}
	\minor{0}{/dev/sbpcd0}{Panasonic CD-ROM controller 0 unit 0}
	\minor{1}{/dev/sbpcd1}{Panasonic CD-ROM controller 0 unit 1}
	\minor{2}{/dev/sbpcd2}{Panasonic CD-ROM controller 0 unit 2}
	\minor{3}{/dev/sbpcd3}{Panasonic CD-ROM controller 0 unit 3}
\end{devicelist}

\begin{devicelist}
\major{26}{}{char }{Quanta WinVision frame grabber}
	\minor{0}{/dev/wvisfgrab}{Quanta WinVision frame grabber}
\\
\major{  }{}{block}{Second Matsushita (Panasonic/SoundBlaster) CD-ROM}
	\minor{0}{/dev/sbpcd4}{Panasonic CD-ROM controller 1 unit 0}
	\minor{1}{/dev/sbpcd5}{Panasonic CD-ROM controller 1 unit 1}
	\minor{2}{/dev/sbpcd6}{Panasonic CD-ROM controller 1 unit 2}
	\minor{3}{/dev/sbpcd7}{Panasonic CD-ROM controller 1 unit 3}
\end{devicelist}

\begin{devicelist}
\major{27}{}{char }{QIC-117 tape}
	\minor{0}{/dev/qft0}{Unit 0, rewind-on-close}
	\minor{1}{/dev/qft1}{Unit 1, rewind-on-close}
	\minor{2}{/dev/qft2}{Unit 2, rewind-on-close}
	\minor{3}{/dev/qft3}{Unit 3, rewind-on-close}
	\minor{4}{/dev/nqft0}{Unit 0, no rewind-on-close}
	\minor{5}{/dev/nqft1}{Unit 1, no rewind-on-close}
	\minor{6}{/dev/nqft2}{Unit 2, no rewind-on-close}
	\minor{7}{/dev/nqft3}{Unit 3, no rewind-on-close}
	\minor{16}{/dev/zqft0}{Unit 0, rewind-on-close, compression}
	\minor{17}{/dev/zqft1}{Unit 1, rewind-on-close, compression}
	\minor{18}{/dev/zqft2}{Unit 2, rewind-on-close, compression}
	\minor{19}{/dev/zqft3}{Unit 3, rewind-on-close, compression}
	\minor{20}{/dev/nzqft0}{Unit 0, no rewind-on-close, compression}
	\minor{21}{/dev/nzqft1}{Unit 1, no rewind-on-close, compression}
	\minor{22}{/dev/nzqft2}{Unit 2, no rewind-on-close, compression}
	\minor{23}{/dev/nzqft3}{Unit 3, no rewind-on-close, compression}
	\minor{32}{/dev/rawqft0}{Unit 0, rewind-on-close, no file marks}
	\minor{33}{/dev/rawqft1}{Unit 1, rewind-on-close, no file marks}
	\minor{34}{/dev/rawqft2}{Unit 2, rewind-on-close, no file marks}
	\minor{35}{/dev/rawqft3}{Unit 3, rewind-on-close, no file marks}
	\minor{36}{/dev/nrawqft0}{Unit 0, no rewind-on-close, no file marks}
	\minor{37}{/dev/nrawqft1}{Unit 1, no rewind-on-close, no file marks}
	\minor{38}{/dev/nrawqft2}{Unit 2, no rewind-on-close, no file marks}
	\minor{39}{/dev/nrawqft3}{Unit 3, no rewind-on-close, no file marks}
\\
\major{  }{}{block}{Third Matsushita (Panasonic/SoundBlaster) CD-ROM}
	\minor{0}{/dev/sbpcd8}{Panasonic CD-ROM controller 2 unit 0}
	\minor{1}{/dev/sbpcd9}{Panasonic CD-ROM controller 2 unit 1}
	\minor{2}{/dev/sbpcd10}{Panasonic CD-ROM controller 2 unit 2}
	\minor{3}{/dev/sbpcd11}{Panasonic CD-ROM controller 2 unit 3}
\end{devicelist}

\begin{devicelist}
\major{28}{}{char }{Stallion serial card -- card programming}
	\minor{0}{/dev/staliomem0}{First Stallion I/O card memory}
	\minor{1}{/dev/staliomem1}{Second Stallion I/O card memory}
	\minor{2}{/dev/staliomem2}{Third Stallion I/O card memory}
	\minor{3}{/dev/staliomem3}{Fourth Stallion I/O card memory}
\\
\major{  }{}{char }{Atari SLM ACSI laser printer (68k/Atari)}
	\minor{0}{/dev/slm0}{First SLM laser printer}
	\minor{1}{/dev/slm1}{Second SLM laser printer}
	\minordots
\\
\major{  }{}{block}{Fourth Matsushita (Panasonic/SoundBlaster) CD-ROM}
	\minor{0}{/dev/sbpcd12}{Panasonic CD-ROM controller 3 unit 0}
	\minor{1}{/dev/sbpcd13}{Panasonic CD-ROM controller 3 unit 1}
	\minor{2}{/dev/sbpcd14}{Panasonic CD-ROM controller 3 unit 2}
	\minor{3}{/dev/sbpcd15}{Panasonic CD-ROM controller 3 unit 3}
\\
\major{  }{}{block}{ACSI disk/CD-ROM (68k/Atari)}
	\minor{0}{/dev/ada}{First ACSI disk whole disk}
	\minor{16}{/dev/adb}{Second ACSI disk whole disk}
	\minor{32}{/dev/adc}{Third ACSI disk whole disk}
	\minordots
	\minor{240}{/dev/adp}{Sixteenth ACSI disk whole disk}
\end{devicelist}

\noindent
Partitions are handled in the same way as for IDE disks (see major
number 3) except that the partition limit is 15 rather than 63 per
disk (same as SCSI.)

\begin{devicelist}
\major{29}{}{char }{Universal frame buffer}
	\minor{0}{/dev/fb0}{First frame buffer}
	\minor{32}{/dev/fb1}{Second frame buffer}
	\minor{64}{/dev/fb2}{Third frame buffer}
	\minordots
	\minor{224}{/dev/fb7}{Eighth frame buffer}
\end{devicelist}

\noindent
All additional minor device numbers are reserved.

\begin{devicelist}
\major{  }{}{block}{Aztech/Orchid/Okano/Wearnes CD-ROM}
	\minor{0}{/dev/aztcd}{Aztech CD-ROM}
\end{devicelist}

\begin{devicelist}
\major{30}{}{char }{iBCS-2 compatibility devices}
	\minor{0}{/dev/socksys}{Socket access}
	\minor{1}{/dev/spx}{SVR3 local X interface}
	\minor{2}{/dev/inet/arp}{Network access}
	\minor{2}{/dev/inet/icmp}{Network access}
	\minor{2}{/dev/inet/ip}{Network access}
	\minor{2}{/dev/inet/udp}{Network access}
	\minor{2}{/dev/inet/tcp}{Network access}
\end{devicelist}

\noindent
Additionally, iBCS-2 requires {\file /dev/nfsd} to be a link to {\file
/dev/socksys} and {\file /dev/X0R} to be a link to {\file /dev/null}.

\begin{devicelist}
\major{  }{}{block}{Philips LMS CM-205 CD-ROM}
	\minor{0}{/dev/cm205cd}{Philips LMS CM-205 CD-ROM}
\end{devicelist}

\noindent
{\file /dev/lmscd} is an older name for this drive.  This driver does
not work with the CM-205MS CD-ROM.

\begin{devicelist}
\major{31}{}{char }{MPU-401 MIDI}
	\minor{0}{/dev/mpu401data}{MPU-401 data port}
	\minor{1}{/dev/mpu401stat}{MPU-401 status port}
\\
\major{  }{}{block}{ROM/flash memory card}
	\minor{0}{/dev/rom0}{First ROM card (rw)}
	\minordots
	\minor{7}{/dev/rom7}{Eighth ROM card (rw)}
	\minor{8}{/dev/rrom0}{First ROM card (ro)}
	\minordots
	\minor{15}{/dev/rrom0}{Eighth ROM card (ro)}
	\minor{16}{/dev/flash0}{First flash memory card (rw)}
	\minordots
	\minor{23}{/dev/flash7}{Eighth flash memory card (rw)}
	\minor{24}{/dev/rflash0}{First flash memory card (ro)}
	\minordots
	\minor{31}{/dev/rflash7}{Eighth flash memory card (ro)}
\end{devicelist}

\noindent
The read-write (rw) devices support back-caching written data in RAM,
as well as writing to flash RAM devices.  The read-only devices (ro)
support reading only.

\begin{devicelist}
\major{32}{}{char }{Specialix serial card}
	\minor{0}{/dev/ttyX0}{First Specialix port}
	\minor{1}{/dev/ttyX1}{Second Specialix port}
	\minordots
\\
\major{  }{}{block}{Philips LMS CM-206 CD-ROM}
	\minor{0}{/dev/cm206cd}{Philips LMS CM-206 CD-ROM}
\end{devicelist}

\begin{devicelist}
\major{33}{}{char }{Specialix serial card -- alternate devices}
	\minor{0}{/dev/cux0}{Callout device corresponding to {\file ttyX0}}
	\minor{1}{/dev/cux1}{Callout device corresponding to {\file ttyX1}}
	\minordots
\\
\major{  }{}{block}{Third IDE hard disk/CD-ROM interface}
	\minor{0}{/dev/hde}{Master: whole disk (or CD-ROM)}
	\minor{64}{/dev/hdf}{Slave: whole disk (or CD-ROM)}
\end{devicelist}

\noindent
Partitions are handled the same way as for the first interface (see
major number 3).

\begin{devicelist}
\major{34}{}{char }{Z8530 HDLC driver}
	\minor{0}{/dev/scc0}{First Z8530, first port}
	\minor{1}{/dev/scc1}{First Z8530, second port}
	\minor{2}{/dev/scc2}{Second Z8530, first port}
	\minor{3}{/dev/scc3}{Second Z8530, second port}
	\minordots
\end{devicelist}

\noindent
In a previous version these devices were named {\file /dev/sc1} for
{\file /dev/scc0}, {\file /dev/sc2} for {\file /dev/scc1}, and so on.

\begin{devicelist}
\major{  }{}{block}{Fourth IDE hard disk/CD-ROM interface}
	\minor{0}{/dev/hdg}{Master: whole disk (or CD-ROM)}
	\minor{64}{/dev/hdh}{Slave: whole disk (or CD-ROM)}
\end{devicelist}

\noindent
Partitions are handled the same way as for the first interface (see
major number 3).

\begin{devicelist}
\major{35}{}{char }{tclmidi MIDI driver}
	\minor{0}{/dev/midi0}{First MIDI port, kernel timed}
	\minor{1}{/dev/midi1}{Second MIDI port, kernel timed}
	\minor{2}{/dev/midi2}{Third MIDI port, kernel timed}
	\minor{3}{/dev/midi3}{Fourth MIDI port, kernel timed}
	\minor{64}{/dev/rmidi0}{First MIDI port, untimed}
	\minor{65}{/dev/rmidi1}{Second MIDI port, untimed}
	\minor{66}{/dev/rmidi2}{Third MIDI port, untimed}
	\minor{67}{/dev/rmidi3}{Fourth MIDI port, untimed}
	\minor{128}{/dev/smpte0}{First MIDI port, SMPTE timed}
	\minor{129}{/dev/smpte1}{Second MIDI port, SMPTE timed}
	\minor{130}{/dev/smpte2}{Third MIDI port, SMPTE timed}
	\minor{131}{/dev/smpte3}{Fourth MIDI port, SMPTE timed}
\\
\major{  }{}{block}{Slow memory ramdisk}
	\minor{0}{/dev/slram}{Slow memory ramdisk}
\end{devicelist}

\begin{devicelist}
\major{36}{}{char }{Netlink support}
	\minor{0}{/dev/route}{Routing, device updates (kernel to user)}
	\minor{1}{/dev/skip}{enSKIP security cache control}
\\
\major{  }{}{block}{MCA ESDI hard disk}
	\minor{0}{/dev/eda}{First ESDI disk whole disk}
	\minor{64}{/dev/edb}{Second ESDI disk whole disk}
	\minordots
\end{devicelist}

\noindent
Partitions are handled the same way as for IDE disks (see major number
3).

\begin{devicelist}
\major{37}{}{char }{IDE tape}
	\minor{0}{/dev/ht0}{First IDE tape}
	\minor{128}{/dev/nht0}{First IDE tape, no rewind-on-close}
\end{devicelist}

\noindent
Currently, only one IDE tape drive is supported.

\begin{devicelist}
\major{  }{}{block}{Zorro II ramdisk}
	\minor{0}{/dev/z2ram}{Zorro II ramdisk}
\end{devicelist}

\begin{devicelist}
\major{38}{}{char }{Myricom PCI Myrinet board}
	\minor{0}{/dev/mlanai0}{First Myrinet board}
	\minor{1}{/dev/mlanai1}{Second Myrinet board}
	\minordots
\end{devicelist}

\noindent
This device is used for board control, status query and ``user level
packet I/O''.  The board is also accessible as a regular {\file eth}
networking device.

\begin{devicelist}
\major{  }{}{block}{Reserved for Linux/AP+}
\end{devicelist}

\begin{devicelist}
\major{39}{}{char }{ML-16P experimental I/O board}
	\minor{0}{/dev/ml16pa-a0}{First card, first analog channel}
	\minor{1}{/dev/ml16pa-a1}{First card, second analog channel}
	\minordots
	\minor{15}{/dev/ml16pa-a15}{First card, 16th analog channel}
	\minor{16}{/dev/ml16pa-d}{First card, digital lines}
	\minor{17}{/dev/ml16pa-c0}{First card, first counter/timer}
	\minor{18}{/dev/ml16pa-c1}{First card, second counter/timer}
	\minor{19}{/dev/ml16pa-c2}{First card, third counter/timer}
	\minor{32}{/dev/ml16pb-a0}{Second card, first analog channel}
	\minor{33}{/dev/ml16pb-a1}{Second card, second analog channel}
	\minordots
	\minor{47}{/dev/ml16pb-a15}{Second card, 16th analog channel}
	\minor{48}{/dev/ml16pb-d}{Second card, digital lines}
	\minor{49}{/dev/ml16pb-c0}{Second card, first counter/timer}
	\minor{50}{/dev/ml16pb-c1}{Second card, second counter/timer}
	\minor{51}{/dev/ml16pb-c2}{Second card, third counter/timer}
	\minordots
\\
\major{  }{}{block}{Reserved for Linux/AP+}
\end{devicelist}

\begin{devicelist}
\major{40}{}{char }{Matrox Meteor frame grabber}
	\minor{0}{/dev/mmetfgrab}{Matrox Meteor frame grabber}
\\
\major{  }{}{block}{Syquest EZ135 parallel port removable drive}
	\minor{0}{/dev/eza}{Parallel EZ135 drive whole disk}
\end{devicelist}

\noindent
This device is obsolete and will be removed in a future version of
Linux.  It has been replaced with the parallel port IDE disk driver at
major number 45.  Partitions are handled the same way as for IDE disks
(see major number 3).

\begin{devicelist}
\major{41}{}{char }{Yet Another Micro Monitor}
	\minor{0}{/dev/yamm}{Yet Another Micro Monitor}
\\
\major{  }{}{block}{MicroSolutions BackPack parallel port CD-ROM}
	\minor{0}{/dev/bpcd}{BackPack CD-ROM}
\end{devicelist}

\noindent
This device is obsolete and will be removed in a future version of
Linux.  It has been replaced with the parallel port ATAPI CD-ROM
driver at major number 46.

\begin{devicelist}
\major{42}{}{}{Demo/sample use}
\end{devicelist}

\noindent
This number is intended for use in sample code, as well as a general
``example'' device number.  It should never be used for a device
driver that is being distributed; either obtain an official number or
use the local/experimental range.  The sudden addition or removal of a
driver with this number should not cause ill effects to the system
(bugs excepted.)

IN PARTICULAR, ANY DISTRIBUTION WHICH CONTAINS A DEVICE DRIVER USING
MAJOR NUMBER 42 IS NONCOMPLIANT.

\begin{devicelist}
\major{43}{}{char }{isdn4linux virtual modem}
	\minor{0}{/dev/ttyI0}{First virtual modem}
	\minordots
	\minor{63}{/dev/ttyI63}{64th virtual modem}
\\
\major{  }{}{block}{Network block devices}
	\minor{0}{/dev/nd0}{First network block device}
	\minor{1}{/dev/nd1}{Second network block device}
	\minordots
\end{devicelist}

\noindent
Network Block Device is somehow similar to loopback devices: If you
read from it, it sends packet accross network asking server for
data. If you write to it, it sends packet telling server to write. It
could be used to mounting filesystems over the net, swapping over the
net, implementing block device in userland etc.

\begin{devicelist}
\major{44}{}{char }{isdn4linux virtual modem -- alternate devices}
	\minor{0}{/dev/cui0}{Callout device corresponding to {\file ttyI0}}
	\minordots
	\minor{63}{/dev/cui63}{Callout device corresponding to {\file ttyI63}}
\\
\major{  }{}{block}{Flash Translation Layer (FTL) filesystems}
        \minor{0}{/dev/ftla}{FTL on first Memory Technology Device}
        \minor{16}{/dev/ftlb}{FTL on second Memory Technology Device}
        \minor{32}{/dev/ftlc}{FTL on third Memory Technology Device}
        \minordots
        \minor{240}{/dev/ftlp}{FTL on 16th Memory Technology Device}
\end{devicelist}

\noindent
Partitions are handled in the same way as for IDE disks (see major
number 3) expect that the partition limit is 15 rather than 63 per
disk (same as SCSI.)

\begin{devicelist}
\major{45}{}{char }{isdn4linux ISDN BRI driver}
	\minor{0}{/dev/isdn0}{First virtual B channel raw data}
	\minordots
	\minor{63}{/dev/isdn63}{64th virtual B channel raw data}
	\minor{64}{/dev/isdnctrl0}{First channel control/debug}
	\minordots
	\minor{127}{/dev/isdnctrl63}{64th channel control/debug}
	\minor{128}{/dev/ippp0}{First SyncPPP device}
	\minordots
	\minor{191}{/dev/ippp63}{64th SyncPPP device}
	\minor{255}{/dev/isdninfo}{ISDN monitor interface}
\\
\major{  }{}{block}{Parallel port IDE disk devices}
	\minor{0}{/dev/pda}{First parallel port IDE disk}
	\minor{16}{/dev/pdb}{Second parallel port IDE disk}
	\minor{32}{/dev/pdc}{Third parallel port IDE disk}
	\minor{48}{/dev/pdd}{Fourth parallel port IDE disk}
\end{devicelist}

\noindent
Partitions are handled in the same way as for IDE disks (see major
number 3) except that the partition limit is 15 rather than 63 per
disk.

\begin{devicelist}
\major{46}{}{char }{Comtrol Rocketport serial card}
	\minor{0}{/dev/ttyR0}{First Rocketport port}
	\minor{1}{/dev/ttyR1}{Second Rocketport port}
	\minordots
\\
\major{  }{}{block}{Parallel port ATAPI CD-ROM devices}
	\minor{0}{/dev/pcd0}{First parallel port ATAPI CD-ROM}
	\minor{1}{/dev/pcd1}{Second parallel port ATAPI CD-ROM}
	\minor{2}{/dev/pcd2}{Third parallel port ATAPI CD-ROM}
	\minor{3}{/dev/pcd3}{Fourth parallel port ATAPI CD-ROM}
\end{devicelist}

\begin{devicelist}
\major{47}{}{char }{Comtrol Rocketport serial card -- alternate devices}
	\minor{0}{/dev/cur0}{Callout device corresponding to {\file ttyR0}}
	\minor{1}{/dev/cur1}{Callout device corresponding to {\file ttyR1}}
	\minordots
\\
\major{  }{}{block}{Parallel port ATAPI disk devices}
	\minor{0}{/dev/pf0}{First parallel port ATAPI disk}
	\minor{1}{/dev/pf1}{Second parallel port ATAPI disk}
	\minor{2}{/dev/pf2}{Third parallel port ATAPI disk}
	\minor{3}{/dev/pf3}{Fourth parallel port ATAPI disk}
\end{devicelist}

\noindent
This driver is intended for floppy disks and similar devices and hence
does not support partitioning.

\begin{devicelist}
\major{48}{}{char }{SDL RISCom serial card}
	\minor{0}{/dev/ttyL0}{First RISCom port}
	\minor{1}{/dev/ttyL1}{Second RISCom port}
	\minordots
\\
\major{  }{}{block}{Reserved for Mylex DAC960 PCI RAID Controller}
\end{devicelist}

\begin{devicelist}
\major{49}{}{char }{SDL RISCom serial card -- alternate devices}
	\minor{0}{/dev/cul0}{Callout device corresponding to {\file ttyL0}}
	\minor{1}{/dev/cul1}{Callout device corresponding to {\file ttyL1}}
	\minordots
\\
\major{  }{}{block}{Reserved for Mylex DAC960 PCI RAID Controller}
\end{devicelist}

\begin{devicelist}
\major{50}{}{char}{Reserved for GLINT}
\\
\major{  }{}{block}{Reserved for Mylex DAC960 PCI RAID Controller}
\end{devicelist}

\begin{devicelist}
\major{51}{}{char }{Baycom radio modem}
	\minor{0}{/dev/bc0}{First Baycom radio modem}
	\minor{1}{/dev/bc1}{Second Baycom radio modem}
	\minordots
\\
\major{  }{}{block}{Reserved for Mylex DAC960 PCI RAID Controller}
\end{devicelist}

\begin{devicelist}
\major{52}{}{char }{Spellcaster DataComm/BRI ISDN card}
	\minor{0}{/dev/dcbri0}{First DataComm card}
	\minor{1}{/dev/dcbri1}{Second DataComm card}
	\minor{2}{/dev/dcbri2}{Third DataComm card}
	\minor{3}{/dev/dcbri3}{Fourth DataComm card}
\\
\major{  }{}{block}{Reserved for Mylex DAC960 PCI RAID Controller}
\end{devicelist}

\begin{devicelist}
\major{53}{}{char }{BDM interface for remote debugging MC683xx microcontrollers}
	\minor{0}{/dev/pd\_bdm0}{PD BDM interface on {\file lp0}}
	\minor{1}{/dev/pd\_bdm1}{PD BDM interface on {\file lp1}}
	\minor{2}{/dev/pd\_bdm2}{PD BDM interface on {\file lp2}}
	\minor{4}{/dev/icd\_bdm0}{ICD BDM interface on {\file lp0}}
	\minor{5}{/dev/icd\_bdm1}{ICD BDM interface on {\file lp1}}
	\minor{6}{/dev/icd\_bdm2}{ICD BDM interface on {\file lp2}}
\end{devicelist}

\noindent
This device is used for the interfacing to the MC683xx
microcontrollers via Background Debug Mode by use of a Parallel Port
interface. PD is the Motorola Public Domain Interface and ICD is the
commercial interface by P\&E.

\begin{devicelist}
\major{  }{}{block}{Reserved for Mylex DAC960 PCI RAID Controller}
\end{devicelist}

\begin{devicelist}
\major{54}{}{char }{Electrocardiognosis Holter serial card}
	\minor{0}{/dev/holter0}{First Holter port}
	\minor{1}{/dev/holter1}{Second Holter port}
	\minor{2}{/dev/holter2}{Third Holter port}
\end{devicelist}

\noindent
A custom serial card used by Electrocardiognosis SRL
$<$mseritan@ottonel.pub.ro$>$ to transfer data from Holter 24-hour
heart monitoring equipment.

\begin{devicelist}
\major{  }{}{block}{Reserved for Mylex DAC960 PCI RAID Controller}
\end{devicelist}

\begin{devicelist}
\major{55}{}{char }{DSP56001 digital signal processor}
	\minor{0}{/dev/dsp56k}{First DSP56001}
\\
\major{  }{}{block}{Reserved for Mylex DAC960 PCI RAID Controller}
\end{devicelist}

\begin{devicelist}
\major{56}{}{char }{Apple Desktop Bus}
	\minor{0}{/dev/adb}{ADB bus control}
\end{devicelist}

\noindent
Additional devices will be added to this number, all starting with
{\file /dev/adb}.

\begin{devicelist}
\major{  }{}{block}{Fifth IDE hard disk/CD-ROM interface}
	\minor{0}{/dev/hdi}{Master: whole disk (or CD-ROM)}
	\minor{64}{/dev/hdj}{Slave: whole disk (or CD-ROM)}
\end{devicelist}

\noindent
Partitions are handled the same way as for the first interface (see
major number 3).

\begin{devicelist}
\major{57}{}{char }{Hayes ESP serial card}
	\minor{0}{/dev/ttyP0}{First ESP port}
	\minor{1}{/dev/ttyP1}{Second ESP port}
	\minordots
\\
\major{  }{}{block}{Sixth IDE hard disk/CD-ROM interface}
	\minor{0}{/dev/hdk}{Master: whole disk (or CD-ROM)}
	\minor{64}{/dev/hdl}{Slave: whole disk (or CD-ROM)}
\end{devicelist}

\noindent
Partitions are handled the same way as for the first interface (see
major number 3).

\begin{devicelist}
\major{58}{}{char }{Hayes ESP serial card -- alternate devices}
	\minor{0}{/dev/cup0}{Callout device corresponding to {\file ttyP0}}
	\minor{1}{/dev/cup1}{Callout device corresponding to {\file ttyP1}}
	\minordots
\\
\major{  }{}{block}{Reserved for logical volume manager}
\end{devicelist}

\begin{devicelist}
\major{59}{}{char }{sf firewall package}
	\minor{0}{/dev/firewall}{Communication with sf kernel module}
\end{devicelist}

\begin{devicelist}
\major{60}{--63}{}{Local/experimental use}
\end{devicelist}

\noindent
For devices not assigned official numbers, these ranges should be
used, in order to avoid conflict with future assignments.

\begin{devicelist}
\major{64}{}{char }{ENskip kernel encryption package}
	\minor{0}{/dev/enskip}{Communication with ENskip kernel
	module}
\end{devicelist}

\begin{devicelist}
\major{65}{}{char }{Sundance ``plink'' Transputer boards}
	\minor{0}{/dev/plink0}{First plink device}
	\minor{1}{/dev/plink1}{Second plink device}
	\minor{2}{/dev/plink2}{Third plink device}
	\minor{3}{/dev/plink3}{Fourth plink device}
	\minor{64}{/dev/rplink0}{First plink device, raw}
	\minor{65}{/dev/rplink1}{Second plink device, raw}
	\minor{66}{/dev/rplink2}{Third plink device, raw}
	\minor{67}{/dev/rplink3}{Fourth plink device, raw}
	\minor{128}{/dev/plink0d}{First plink device, debug}
	\minor{129}{/dev/plink1d}{Second plink device, debug}
	\minor{130}{/dev/plink2d}{Third plink device, debug}
	\minor{131}{/dev/plink3d}{Fourth plink device, debug}
	\minor{192}{/dev/rplink0d}{First plink device, raw, debug}
	\minor{193}{/dev/rplink1d}{Second plink device, raw, debug}
	\minor{194}{/dev/rplink2d}{Third plink device, raw, debug}
	\minor{195}{/dev/rplink3d}{Fourth plink device, raw, debug}
\end{devicelist}

\noindent
This is a commercial driver; contact James Howes
$<$jth@prosig.demon.co.uk$>$ for information. 

\begin{devicelist}
\major{  }{}{block}{SCSI disk devices (16-31)}
	\minor{0}{/dev/sdq}{17th SCSI disk whole disk}
	\minor{16}{/dev/sdr}{18th SCSI disk whole disk}
	\minor{32}{/dev/sds}{19th SCSI disk whole disk}
	\minordots
	\minor{240}{/dev/sdaf}{32nd SCSI disk whole disk}
\end{devicelist}

\noindent
Partitions are handled in the same way as for IDE disks (see major
number 3) except that the partition limit is 15 rather than 63 per
disk.

\begin{devicelist}
\major{66}{}{char }{YARC PowerPC PCI coprocessor card}
	\minor{0}{/dev/yppcpci0}{First YARC card}
	\minor{1}{/dev/yppcpci1}{Second YARC card}
	\minordots
\end{devicelist}

\begin{devicelist}
\major{  }{}{block}{SCSI disk devices (32-47)}
	\minor{0}{/dev/sdag}{33rd SCSI disk whole disk}
	\minor{16}{/dev/sdah}{34th SCSI disk whole disk}
	\minor{32}{/dev/sdai}{35th SCSI disk whole disk}
	\minordots
	\minor{240}{/dev/sdav}{48th SCSI disk whole disk}
\end{devicelist}

\noindent
Partitions are handled in the same way as for IDE disks (see major
number 3) except that the partition limit is 15 rather than 63 per
disk.

\begin{devicelist}
\major{67}{}{char }{Coda network filesystem}
	\minor{0}{/dev/cfs0}{Coda cache manager}
\end{devicelist}

\noindent
See {\url http://www.coda.cs.cmu.edu\/} for information about Coda.

\begin{devicelist}
\major{  }{}{block}{SCSI disk devices (48-63)}
	\minor{0}{/dev/sdaw}{49th SCSI disk whole disk}
	\minor{16}{/dev/sdax}{50th SCSI disk whole disk}
	\minor{32}{/dev/sday}{51st SCSI disk whole disk}
	\minordots
	\minor{240}{/dev/sdbl}{64th SCSI disk whole disk}
\end{devicelist}

\noindent
Partitions are handled in the same way as for IDE disks (see major
number 3) except that the partition limit is 15 rather than 63 per
disk.

\begin{devicelist}
\major{68}{}{char }{CAPI 2.0 interface}
	\minor{0}{/dev/capi20}{Control device}
	\minor{1}{/dev/capi20.00}{First CAPI 2.0 application}
	\minor{2}{/dev/capi20.01}{Second CAPI 2.0 application}
	\minordots
	\minor{20}{/dev/capi20.19}{19th CAPI 2.0 application}
\end{devicelist}

\noindent
ISDN CAPI 2.0 driver for use with CAPI 2.0 applications; currently
supports the AVM B1 card.

\begin{devicelist}
\major{  }{}{block}{SCSI disk devices (64-79)}
	\minor{0}{/dev/sdbm}{65th SCSI disk whole disk}
	\minor{16}{/dev/sdbn}{66th SCSI disk whole disk}
	\minor{32}{/dev/sdbo}{67th SCSI disk whole disk}
	\minordots
	\minor{240}{/dev/sdcb}{80th SCSI disk whole disk}
\end{devicelist}

\noindent
Partitions are handled in the same way as for IDE disks (see major
number 3) except that the partition limit is 15 rather than 63 per
disk.

\begin{devicelist}
\major{69}{}{char }{MA16 numeric accelerator card}
	\minor{0}{/dev/ma16}{Board memory access}
\end{devicelist}

\begin{devicelist}
\major{  }{}{block}{SCSI disk devices (80-95)}
	\minor{0}{/dev/sdcc}{81st SCSI disk whole disk}
	\minor{16}{/dev/sdcd}{82nd SCSI disk whole disk}
	\minor{32}{/dev/sdce}{83th SCSI disk whole disk}
	\minordots
	\minor{240}{/dev/sdcr}{96th SCSI disk whole disk}
\end{devicelist}

\noindent
Partitions are handled in the same way as for IDE disks (see major
number 3) except that the partition limit is 15 rather than 63 per
disk.

\begin{devicelist}
\major{70}{}{char }{SpellCaster Protocol Services Interface}
	\minor{0}{/dev/apscfg}{Configuration interface}
	\minor{1}{/dev/apsauth}{Authentication interface}
	\minor{2}{/dev/apslog}{Logging interface}
	\minor{3}{/dev/apsdbg}{Debugging interface}
	\minor{64}{/dev/apsisdn}{ISDN command interface}
	\minor{65}{/dev/apsasync}{Async command interface}
	\minor{128}{/dev/apsmon}{Monitor interface}
\end{devicelist}

\begin{devicelist}
\major{  }{}{block}{SCSI disk devices (96-111)}
	\minor{0}{/dev/sdcs}{97th SCSI disk whole disk}
	\minor{16}{/dev/sdct}{98th SCSI disk whole disk}
	\minor{32}{/dev/sdcu}{99th SCSI disk whole disk}
	\minordots
	\minor{240}{/dev/sddh}{112th SCSI disk whole disk}
\end{devicelist}

\noindent
Partitions are handled in the same way as for IDE disks (see major
number 3) except that the partition limit is 15 rather than 63 per
disk.

\begin{devicelist}
\major{71}{}{char }{Computone IntelliPort II serial card}
	\minor{0}{/dev/ttyF0}{IntelliPort II board 0, port 0}
	\minor{1}{/dev/ttyF1}{IntelliPort II board 0, port 1}
	\minordots
	\minor{63}{/dev/ttyF63}{IntelliPort II board 0, port 63}
	\minor{64}{/dev/ttyF64}{IntelliPort II board 1, port 0}
	\minor{65}{/dev/ttyF65}{IntelliPort II board 1, port 1}
	\minordots
	\minor{127}{/dev/ttyF127}{IntelliPort II board 1, port 63}
	\minor{128}{/dev/ttyF128}{IntelliPort II board 2, port 0}
	\minor{129}{/dev/ttyF129}{IntelliPort II board 2, port 1}
	\minordots
	\minor{191}{/dev/ttyF191}{IntelliPort II board 2, port 63}
	\minor{192}{/dev/ttyF192}{IntelliPort II board 3, port 0}
	\minor{193}{/dev/ttyF193}{IntelliPort II board 3, port 1}
	\minordots
	\minor{255}{/dev/ttyF255}{IntelliPort II board 3, port 63}
\end{devicelist}

\begin{devicelist}
\major{  }{}{block}{SCSI disk devices (112-127)}
	\minor{0}{/dev/sddi}{113th SCSI disk whole disk}
	\minor{16}{/dev/sddj}{114th SCSI disk whole disk}
	\minor{32}{/dev/sddk}{115th SCSI disk whole disk}
	\minordots
	\minor{240}{/dev/sddx}{128th SCSI disk whole disk}
\end{devicelist}

\noindent
Partitions are handled in the same way as for IDE disks (see major
number 3) except that the partition limit is 15 rather than 63 per
disk.

\begin{devicelist}
\major{72}{}{char }{Computone IntelliPort II serial card -- alternate devices}
	\minor{0}{/dev/cuf0}{Callout device corresponding to {\file ttyF0}}
	\minor{1}{/dev/cuf1}{Callout device corresponding to {\file ttyF1}}
	\minordots
	\minor{63}{/dev/cuf63}{Callout device corresponding to {\file ttyF63}}
	\minor{64}{/dev/cuf64}{Callout device corresponding to {\file ttyF64}}
	\minor{65}{/dev/cuf65}{Callout device corresponding to {\file ttyF65}}
	\minordots
	\minor{127}{/dev/cuf127}{Callout device corresponding to {\file ttyF127}}
	\minor{128}{/dev/cuf128}{Callout device corresponding to {\file ttyF128}}
	\minor{129}{/dev/cuf129}{Callout device corresponding to {\file ttyF129}}
	\minordots
	\minor{191}{/dev/cuf191}{Callout device corresponding to {\file ttyF191}}
	\minor{192}{/dev/cuf192}{Callout device corresponding to {\file ttyF192}}
	\minor{193}{/dev/cuf193}{Callout device corresponding to {\file ttyF193}}
	\minordots
	\minor{255}{/dev/cuf255}{Callout device corresponding to {\file ttyF255}}
\end{devicelist}

\begin{devicelist}
\major{73}{}{char }{Computone IntelliPort II serial card -- control devices}
	\minor{0}{/dev/ip2ipl0}{Loadware device for board 0}
	\minor{1}{/dev/ip2stat0}{Status device for board 0}
	\minor{4}{/dev/ip2ipl1}{Loadware device for board 1}
	\minor{5}{/dev/ip2stat1}{Status device for board 1}
	\minor{8}{/dev/ip2ipl2}{Loadware device for board 2}
	\minor{9}{/dev/ip2stat2}{Status device for board 2}
	\minor{12}{/dev/ip2ipl3}{Loadware device for board 3}
	\minor{13}{/dev/ip2stat3}{Status device for board 3}
\end{devicelist}

\begin{devicelist}
\major{74}{}{char }{SCI bridge}
	\minor{0}{/dev/SCI/0}{SCI device 0}
	\minor{1}{/dev/SCI/1}{SCI device 1}
	\minordots
\end{devicelist}

\noindent
Currently for Dolphin Interconnect Solutions' PCI-SCI bridge.

\begin{devicelist}
\major{75}{}{char }{Specialix IO8+ serial card}
	\minor{0}{/dev/ttyW0}{First IO8+ port, first card}
	\minor{1}{/dev/ttyW1}{Second IO8+ port, first card}
	\minordots
	\minor{8}{/dev/ttyW8}{First IO8+ port, second card}
	\minordots
\end{devicelist}

\begin{devicelist}
\major{76}{}{char }{Specialix IO8+ serial card -- alternate devices}
	\minor{0}{/dev/cuw0}{Callout device corresponding to {\file ttyW0}}
	\minor{1}{/dev/cuw1}{Callout device corresponding to {\file ttyW1}}
	\minordots
	\minor{8}{/dev/cuw8}{Callout device corresponding to {\file ttyW8}}
	\minordots
\end{devicelist}

\begin{devicelist}
\major{77}{}{char }{ComScire Quantum Noise Generator}
	\minor{0}{/dev/qng}{ComScire Quantum Noise Generator}
\end{devicelist}

\begin{devicelist}
\major{78}{}{char }{PAM Software's multimodem boards}
	\minor{0}{/dev/ttyM0}{First PAM modem}
	\minor{1}{/dev/ttyM1}{Second PAM modem}
	\minordots
\end{devicelist}

\begin{devicelist}
\major{79}{}{char }{PAM Software's multimodem boards -- alternate devices}
	\minor{0}{/dev/cum0}{Callout device corresponding to {\file ttyM0}}
	\minor{1}{/dev/cum1}{Callout device corresponding to {\file ttyM1}}
	\minordots
\end{devicelist}

\begin{devicelist}
\major{80}{}{char }{Photometrics AT200 CCD camera}
	\minor{0}{/dev/at200}{Photometrics AT200 CCD camera}
\end{devicelist}

\begin{devicelist}
\major{81}{}{char }{video4linux}
	\minor{0}{/dev/video0}{Video capture/overlay device}
	\minordots
	\minor{63}{/dev/video63}{Video capture/overlay device}
	\minor{64}{/dev/radio0}{Radio device}
	\minordots
	\minor{127}{/dev/radio63}{Radio device}
	\minor{192}{/dev/vtx0}{Teletext device}
	\minordots
	\minor{223}{/dev/vtx31}{Teletext device}
	\minor{224}{/dev/vbi0}{Vertical blank interupt}
	\minordots
	\minor{255}{/dev/vbi31}{Vertical blank interupt}
\end{devicelist}

\begin{devicelist}
\major{82}{}{char }{WiNRADiO communications receiver card}
	\minor{0}{/dev/winradio0}{First WiNRADiO card}
	\minor{1}{/dev/winradio1}{Second WiNRADiO card}
	\minordots
\end{devicelist}

\noindent
The driver and documentation may be obtained from
{\url http://www.proximity.com.au/~brian/winradio/\/}.

\begin{devicelist}
\major{83}{}{char }{Teletext/videotext interfaces}
        \minor{0}{/dev/vtx}{Teletext decoder}
        \minor{16}{/dev/vttuner}{TV tuner on teletext interface}
\end{devicelist}

\noindent
Devices for the driver contained in the VideoteXt package. More information
on {\url http://home.pages.de/~videotext/\/}.

\begin{devicelist}
\major{84}{}{char }{Ikon 1011[57] Versatec Greensheet Interface}
	\minor{0}{/dev/ihcp0}{First Greensheet port}
	\minor{1}{/dev/ihcp1}{Second Greensheet port}
\end{devicelist}

\begin{devicelist}
\major{85}{}{char }{Linux/SGI shared memory input queue}
	\minor{0}{/dev/shmiq}{Master shared input queue}
	\minor{1}{/dev/qcntl0}{First device pushed}
	\minor{2}{/dev/qcntl1}{Second device pushed}
	\minordots
\end{devicelist}

\begin{devicelist}
\major{86}{}{char }{SCSI media changer}
	\minor{0}{/dev/sch0}{First SCSI media changer}
	\minor{1}{/dev/sch1}{Second SCSI media changer}
	\minordots
\end{devicelist}

\begin{devicelist}
\major{87}{}{char }{Sony Control-A1 stereo control bus}
	\minor{0}{/dev/controla0}{First device on chain}
	\minor{1}{/dev/controla1}{Second device on chain}
	\minordots
\end{devicelist}

\begin{devicelist}
\major{88}{}{char }{COMX synchronous serial card}
	\minor{0}{/dev/comx0}{Channel 0}
	\minor{1}{/dev/comx1}{Channel 1}
	\minordots
\end{devicelist}

\begin{devicelist}
\major{89}{}{char }{I$^2$C bus interface}
	\minor{0}{/dev/i2c0}{First I$^2$C adapter}
	\minor{1}{/dev/i2c1}{Second I$^2$C adapter}
	\minordots
\end{devicelist}

\begin{devicelist}
\major{90}{}{char }{Memory Technology Device (RAM, ROM, Flash)}
        \minor{0}{/dev/mtd0}{First MTD (rw)}
        \minor{1}{/dev/mtdr0}{First MTD (ro)}
        \minordots
        \minor{30}{/dev/mtd15}{16th MTD (rw)}
        \minor{31}{/dev/mtdr15}{16th MTD (ro)}
\end{devicelist}

\begin{devicelist}
\major{91}{}{char }{CAN-Bus controller}
	\minor{0}{/dev/can0}{First CAN-Bus controller}
	\minor{1}{/dev/can1}{Second CAN-Bus controller}
	\minordots
\end{devicelist}

\begin{devicelist}
\major{92}{}{char }{Reserved for ith Kommunikationstechnik MIC ISDN card}
\end{devicelist}

\begin{devicelist}
\major{93}{}{char }{IBM Smart Capture Card frame grabber}
	\minor{0}{/dev/iscc0}{First Smart Capture Card}
	\minor{1}{/dev/iscc1}{Second Smart Capture Card}
	\minordots
	\minor{128}{/dev/isccctl0}{First Smart Capture Card control}
	\minor{129}{/dev/isccctl1}{Second Smart Capture Card control}
	\minordots
\end{devicelist}

\begin{devicelist}
\major{94}{}{char }{miroVIDEO DC10/30 capture/playback device}
	\minor{0}{/dev/dcxx0}{First capture card}
	\minor{1}{/dev/dcxx1}{Second capture card}
	\minordots
\end{devicelist}

\begin{devicelist}
\major{95}{}{char }{IP filter}
	\minor{0}{/dev/ipl}{Filter control device/log file}
	\minor{1}{/dev/ipnat}{NAT control device/log file}
	\minor{2}{/dev/ipstate}{State information log file}
	\minor{3}{/dev/ipauth}{Authentication control device/log file}
\end{devicelist}

\begin{devicelist}
\major{96}{}{char }{Parallel port ATAPI tape devices}
	\minor{0}{/dev/pt0}{First parallel port ATAPI tape}
	\minor{1}{/dev/pt1}{Second parallel port ATAPI tape}
	\minor{2}{/dev/pt2}{Third parallel port ATAPI tape}
	\minor{3}{/dev/pt3}{Fourth parallel port ATAPI tape}
	\minor{128}{/dev/npt0}{First parallel port ATAPI tape, no rewind}
	\minor{129}{/dev/npt1}{Second parallel port ATAPI tape, no rewind}
	\minor{130}{/dev/npt2}{Third parallel port ATAPI tape, no rewind}
	\minor{131}{/dev/npt3}{Fourth parallel port ATAPI tape, no rewind}
\end{devicelist}

\begin{devicelist}
\major{97}{}{char }{Parallel port generic ATAPI interface}
	\minor{0}{/dev/pg0}{First parallel port ATAPI device}
	\minor{1}{/dev/pg1}{Second parallel port ATAPI device}
	\minor{2}{/dev/pg2}{Third parallel port ATAPI device}
	\minor{3}{/dev/pg3}{Fourth parallel port ATAPI device}
\end{devicelist}

\noindent
These devices support the same API as the generic SCSI devices.

\begin{devicelist}
\major{98}{}{char }{Control and Mesurement Device (comedi)}
	\minor{0}{/dev/comedi0}{First comedi device}
	\minor{1}{/dev/comedi1}{Second comedi device}
	\minordots
\end{devicelist}

\noindent
See {\url http://stm.lbl.gov/comedi/} or {\url
http://www.llp.fu-berlin.de/}.

\begin{devicelist}
\major{99}{}{char }{Raw parallel ports}
	\minor{0}{/dev/parport0}{First parallel port}
	\minor{1}{/dev/parport1}{Second parallel port}
	\minordots
\end{devicelist}

\noindent
These devices can be used to drive parallel port devices from
user-space while interacting with the parport sharing code.

\begin{devicelist}
\major{100}{}{char }{POTS (analogue telephone) A/B port}
	\minor{0}{/dev/phone0}{First telephone port}
	\minor{1}{/dev/phone1}{Second telephone port}
	\minordots
\end{devicelist}

\begin{devicelist}
\major{101}{}{char }{Motorola DSP 56xxx board}
	\minor{0}{/dev/mdspstat}{Status information}
	\minor{1}{/dev/mdsp1}{First DSP board I/O and controls}
	\minordots
	\minor{16}{/dev/mdsp16}{16th DSP board I/O and controls}
\end{devicelist}

\begin{devicelist}
\major{102}{}{char }{Philips SAA5249 Teletext signal decoder}
	\minor{0}{/dev/tlk0}{First Teletext decoder}
	\minor{1}{/dev/tlk1}{Second Teletext decoder}
	\minor{2}{/dev/tlk2}{Third Teletext decoder}
	\minor{3}{/dev/tlk3}{Fourth Teletext decoder}
\end{devicelist}

\begin{devicelist}
\major{103}{}{char }{Arla network file system}
	\minor{0}{/dev/xfs0}{Arla XFS}
\end{devicelist}

\noindent
Arla is a free clone of the Andrew File System, AFS.  Any resemblance
with the Swedish milk producer is coincidental.  For more information
about the project, write to $<$arla-drinkers@stacken.kth.se$>$ or
subscribe to the arla-announce mailing list by sending a mail to
$<$arla-announce-request@stacken.kth.se$>$.

\begin{devicelist}
\major{104}{}{char }{Flash BIOS support}
\end{devicelist}

\begin{devicelist}
\major{105}{}{char }{Comtrol VS-1000 serial card}
	\minor{0}{/dev/ttyV0}{First VS-1000 port}
	\minor{1}{/dev/ttyV1}{Second VS-1000 port}
	\minordots
\end{devicelist}

\begin{devicelist}
\major{106}{}{char }{Comtrol VS-1000 serial card -- alternate devices}
	\minor{0}{/dev/cuv0}{Callout device corresponding to {\file ttyV0}}
	\minor{1}{/dev/cuv1}{Callout device corresponding to {\file ttyV1}}
	\minordots
\end{devicelist}

\begin{devicelist}
\major{107}{}{char }{3Dfx Voodoo Graphics device}
	\minor{0}{/dev/3dfx}{Primary 3Dfx graphics device}
\end{devicelist}

\begin{devicelist}
\major{108}{}{char }{Device independent PPP interface}
	\minor{0}{/dev/ppp}{Device independent PPP interface}
\end{devicelist}

\begin{devicelist}
\major{109}{}{char }{Reserved for logical volume manager}
\end{devicelist}

\begin{devicelist}
\major{110}{}{char }{miroMEDIA Surround board}
	\minor{0}{/dev/srnd0}{First miroMEDIA Surround board}
	\minor{1}{/dev/srnd1}{First miroMEDIA Surround board}
	\minordots
\end{devicelist}

\begin{devicelist}
\major{111}{--119}{}{Unallocated}
\end{devicelist}

\begin{devicelist}
\major{120}{--127}{}{Local/experimental use}
\end{devicelist}

\begin{devicelist}
\major{128}{--135}{char }{Unix98 PTY masters}
\end{devicelist}

\noindent
These devices should not have corresponding device nodes; instead they
should be accessed through the {\file /dev/ptmx} cloning device.

\begin{devicelist}
\major{136}{--143}{char }{Unix98 PTY slaves}
	\minor{0}{/dev/pts/0}{First Unix98 pseudo-TTY}
	\minor{1}{/dev/pts/1}{Second Unix98 pseudo-TTY}
	\minordots
\end{devicelist}

\noindent
These device nodes are automatically generated with the proper
permissions and modes by mounting the {\file devpts} filesystem onto
{\file /dev/pts} with the appropriate mount options (distribution
dependent.)

\begin{devicelist}
\major{144}{--239}{}{Unallocated}
\end{devicelist}

\begin{devicelist}
\major{240}{--254}{}{Local/experimental use}
\end{devicelist}

\begin{devicelist}
\major{255}{}{}{Reserved}
\end{devicelist}

\noindent
This major is reserved to assist the expansion to a larger number
space.  No device nodes with this major should ever be created on any
filesystem.

\section{Additional /dev directory entries}

This section details additional entries that should or may exist in the
{\file /dev} directory.  It is preferred that symbolic links use the
same form (absolute or relative) as is indicated here.  Links are
classified as {\em hard\/} or {\em symbolic\/} depending on the
preferred type of link; if possible, the indicated type of link should
be used.

\subsection{Compulsory links}

These links should exist on all systems:

\begin{nodelist}
\link{/dev/fd}{/proc/self/fd}{symbolic}{File descriptors}
\link{/dev/stdin}{fd/0}{symbolic}{Standard input file descriptor}
\link{/dev/stdout}{fd/1}{symbolic}{Standard output file descriptor}
\link{/dev/stderr}{fd/2}{symbolic}{Standard error file descriptor}
\link{/dev/nfsd}{socksys}{symbolic}{Required by iBCS-2}
\link{/dev/X0R}{null}{symbolic}{Required by iBCS-2}
\end{nodelist}

\noindent
Note: The last device is: $<$letter {\tt X}$>$-$<$digit {\tt
0}$>$-$<$letter {\tt R}$>$.

\subsection{Recommended links}

It is recommended that these links exist on all systems:

\begin{nodelist}
\link{/dev/core}{/proc/kcore}{symbolic}{Backward compatibility}
\link{/dev/ramdisk}{ram0}{symbolic}{Backward compatibility}
\link{/dev/ftape}{qft0}{symbolic}{Backward compatibility}
\link{/dev/bttv0}{video0}{symbolic}{Backward compatibility}
\link{/dev/radio}{radio0}{symbolic}{Backward compatibility}
\link{/dev/scd?}{sr?}{hard}{Alternate name for CD-ROMs}
\end{nodelist}

\subsection{Locally defined links}

The following links may be established locally to conform to the
configuration of the system.  This is merely a tabulation of existing
practice, and does not constitute a recommendation.  However, if they
exist, they should have the following uses.

\begin{nodelist}
\vlink{/dev/mouse}{mouse port}{symbolic}{Current mouse device}
\vlink{/dev/tape}{tape device}{symbolic}{Current tape device}
\vlink{/dev/cdrom}{CD-ROM device}{symbolic}{Current CD-ROM device}
\vlink{/dev/cdwriter}{CD-writer}{symbolic}{Current CD-writer device}
\vlink{/dev/scanner}{scanner device}{symbolic}{Current scanner device}
\vlink{/dev/modem}{modem port}{symbolic}{Current dialout device}
\vlink{/dev/root}{root device}{symbolic}{Current root filesystem}
\vlink{/dev/swap}{swap device}{symbolic}{Current swap device}
\end{nodelist}

\noindent
{\file /dev/modem} should not be used for a modem which supports
dialin as well as dialout, as it tends to cause lock file problems.
If it exists, {\file /dev/modem} should point to the appropriate
primary TTY device (the use of the alternate callout devices is
deprecated.)

For SCSI devices, {\file /dev/tape} and {\file /dev/cdrom} should
point to the ``cooked'' devices ({\file /dev/st*} and {\file
/dev/sr*}, respectively), whereas {\file /dev/cdwriter} and {\file
/dev/scanner} should point to the appropriate generic SCSI devices
({\file /dev/sg*}).

{\file /dev/mouse} may point to a primary serial TTY device, a
hardware mouse device, or a socket for a mouse driver program
(e.g. {\file /dev/gpmdata}).

\subsection{Sockets and pipes}

Non-transient sockets or named pipes may exist in {\file /dev}.
Common entries are:

\begin{nodelist}
\node{/dev/printer}{socket}{{\file lpd} local socket}
\node{/dev/log}{socket}{{\file syslog} local socket}
\node{/dev/gpmdata}{socket}{{\file gpm} mouse multiplexer}
\end{nodelist}

\section{Terminal devices}

Terminal, or TTY devices are a special class of character devices.  A
terminal device is any device that could act as a controlling terminal
for a session; this includes virtual consoles, serial ports, and
pseudoterminals (PTYs).

All terminal devices share a common set of capabilities known as line
diciplines; these include the common terminal line dicipline as well
as SLIP and PPP modes.

All terminal devices are named similarly; this section explains the
naming and use of the various types of TTYs.  Note that the naming
conventions include several historical warts; some of these are
Linux-specific, some were inherited from other systems, and some
reflect Linux outgrowing a borrowed convention.

A hash mark ($\#$) in a device name is in all cases used here to
indicate a decimal number without leading zeroes.

\subsection{Virtual consoles and the console device}

Virtual consoles are full-screen terminal displays on the system video
monitor.  Virtual consoles are named {\file /dev/tty$\#$}, with
numbering starting at {\file /dev/tty1}; {\file /dev/tty0} is the
current virtual console.  {\file /dev/tty0} is the device that should
be used to access the system video card on those architectures for
which the frame buffer devices ({\file /dev/fb*}) are not applicable.
Do not use {\file /dev/console} for this purpose.

The {\em console device\/}, {\file /dev/console}, is the device to
which system messages should be sent, and on which logins should be
permitted in single-user mode.  Starting with Linux 2.1.71, {\file
/dev/console} is managed by the kernel; for previous versions it
should be a symbolic link to either {\file /dev/tty0}, a specific
virtual console such as {\file /dev/tty1}, or to a serial port primary
({\file tty*}, not {\file cu*}) device, depending on the configuration
of the system.

\subsection{Serial ports}

Serial ports are RS-232 serial ports and any device which simulates
one, either in hardware (such as internal modems) or in software (such
as the ISDN driver.)  Under Linux, each serial ports has two device
names, the primary or callin device and the alternate or callout one.
Each kind of device is indicated by a different letter.  For any
letter $X$, the names of the devices are {\file /dev/tty${X\#}$} and
{\file /dev/cu${x\#}$}, respectively; for historical reasons, {\file
/dev/ttyS$\#$} and {\file /dev/ttyC$\#$} correspond to {\file
/dev/cua$\#$} and {\file /dev/cub$\#$}.  In the future, it should be
expected that multiple letters will be used; all letters will be upper
case for the {\file tty} device (e.g. {\file /dev/ttyDP$\#$} and lower
case for the {\file cu} device (e.g. {\file /dev/cudp$\#$}.

The use of the callout devices is deprecated.

The names {\file /dev/ttyQ$\#$} and {\file /dev/cuq$\#$} are reserved
for local use.

The alternate devices provide for kernel-based exclusion and somewhat
different defaults than the primary devices.  Their main purpose is to
allow the use of serial ports with programs with no inherent or broken
support for serial ports.  Their use is deprecated, and they may be
removed from a future version of Linux.

Arbitration of serial ports is provided by the use of lock files with
the names {\file /var/lock/LCK..tty${X\#}$}.  The contents of the lock
file should be the PID of the locking process as an ASCII number.

It is common practice to install links such as {\file /dev/modem\/}
which point to serial ports.  In order to ensure proper locking in the
presence of these links, it is recommended that software chase
symlinks and lock all possible names; additionally, it is recommended
that a lock file be installed with the corresponding alternate
device.  In order to avoid deadlocks, it is recommended that the locks
are acquired in the following order, and released in the reverse:
\begin{itemize}
\item{The symbolic link name, if any ({\file /var/lock/LCK..modem})}
\item{The {\file tty} name ({\file /var/lock/LCK..ttyS2})}
\item{The alternate device name ({\file /var/lock/LCK..cua2})}
\end{itemize}
In the case of nested symbolic links, the lock files should be
installed in the order the symlinks are resolved.

Under no circumstances should an application hold a lock while waiting
for another to be released.  In addition, applications which attempt
to create lock files for the corresponding alternate device names
should take into account the possibility of being used on a non-serial
port TTY, for which no alternate device would exist.

\subsection{Pseudoterminals (PTYs)}

Pseudoterminals, or PTYs, are used to create login sessions or provide
other capabilities requiring a TTY line dicipline (including SLIP or
PPP capability) to arbitrary data-generation processes.  Each PTY has
a {\em master\/} side, named {\file /dev/pty[p-za-e][0-9a-f]\/}, and a
{\em slave\/} side, named {\file /dev/tty[p-za-e][0-9a-f]\/}.  The
kernel arbitrates the use of PTYs by allowing each master side to be
opened only once.

Once the master side has been opened, the corresponding slave device
can be used in the same manner as any TTY device.  The master and
slave devices are connected by the kernel, generating the equivalent
of a bidirectional pipe with TTY capabilities.

Recent versions of the Linux kernels and GNU libc contain support for
the System V/Unix98 naming scheme for PTYs, which assigns a common
device {\file /dev/ptmx\/} to all the masters (opening it will
automatically give you a previously unassigned PTY) and a subdirectory
{\file /dev/pts\/} for the slaves; the slaves are named with decimal
integers ({\file /dev/pts/$\#$\/} in our notation).  This removes the
problem of exhausting the namespace and enables the kernel to
automatically create the device nodes for the slaves on demand using
the {\file devpts\/} filesystem.

\end{document}
